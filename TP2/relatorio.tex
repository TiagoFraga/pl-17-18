%
% Layout retirado de http://www.di.uminho.pt/~prh/curplc09.html#notas
%
\documentclass{report}
\usepackage[utf8]{inputenc}
\usepackage[latin1]{inputenc}
\usepackage[portuguese]{babel}

\usepackage{url}
\usepackage{enumerate}
\usepackage{fancyvrb}
\usepackage{amsmath}

%\usepackage{alltt}
%\usepackage{fancyvrb}
\usepackage{listings}
%LISTING - GENERAL
\lstset{
	basicstyle=\small,
	numbers=left,
	numberstyle=\tiny,
	numbersep=5pt,
	breaklines=true,
    frame=tB,
	mathescape=true,
	escapeinside={(*@}{@*)}
}
%
%\lstset{ %
%	language=Java,							% choose the language of the code
%	basicstyle=\ttfamily\footnotesize,		% the size of the fonts that are used for the code
%	keywordstyle=\bfseries,					% set the keyword style
%	%numbers=left,							% where to put the line-numbers
%	numberstyle=\scriptsize,				% the size of the fonts that are used for the line-numbers
%	stepnumber=2,							% the step between two line-numbers. If it's 1 each line
%											% will be numbered
%	numbersep=5pt,							% how far the line-numbers are from the code
%	backgroundcolor=\color{white},			% choose the background color. You must add \usepackage{color}
%	showspaces=false,						% show spaces adding particular underscores
%	showstringspaces=false,					% underline spaces within strings
%	showtabs=false,							% show tabs within strings adding particular underscores
%	frame=none,								% adds a frame around the code
%	%abovecaptionskip=-.8em,
%	%belowcaptionskip=.7em,
%	tabsize=2,								% sets default tabsize to 2 spaces
%	captionpos=b,							% sets the caption-position to bottom
%	breaklines=true,						% sets automatic line breaking
%	breakatwhitespace=false,				% sets if automatic breaks should only happen at whitespace
%	title=\lstname,							% show the filename of files included with \lstinputlisting;
%											% also try caption instead of title
%	escapeinside={\%*}{*)},					% if you want to add a comment within your code
%	morekeywords={*,...}					% if you want to add more keywords to the set
%}

\usepackage{xspace}

\setlength{\parindent}{3em}
\setlength{\parskip}{1em}

\setlength{\oddsidemargin}{-1cm}
\setlength{\textwidth}{18cm}
\setlength{\headsep}{-1cm}
\setlength{\textheight}{23cm}

\def\darius{\textsf{Darius}\xspace}
\def\antlr{\texttt{AnTLR}\xspace}
\def\pe{\emph{Publicação Eletrónica}\xspace}

\def\titulo#1{\section{#1}}
\def\super#1{{\em Supervisor: #1}\\ }
\def\area#1{{\em \'{A}rea: #1}\\[0.2cm]}
\def\resumo{\underline{Resumo}:\\ }



\title{Processamento de Linguagens\\ \textbf{Trabalho Prático nº2 (Flex)}\\ \textbf{XML to Dot} \\Relatório de Desenvolvimento}

\author{Cesário Perneta\\ a73883 \and João Gomes\\ a74033 \and Tiago Fraga\\ a74092}
\date{\today}

\begin{document}

\DefineVerbatimEnvironment{verbatim}{Verbatim}{xleftmargin=.5in}

\maketitle

\begin{abstract}
    Neste relatório será abordado a resolução e verificação do segundo trabalho prático da unidade curricular de Processamento de Linguagens. Será possível encontrar a forma como abordamos o problema proposto bem como uma explicação detalhada do código elaborado.
\end{abstract}

\tableofcontents

\chapter{ XML to Dot} \label{intro}
    
    O nosso grupo teve como enunciado o \textit{XML to Dot}, neste enunciado foi nos proposto que fizessemos uma análise aos ficheiros \textit{.xml} fornecidos. 
    Depois de efetuada esta análise, foi nos pedido que desenvolvessemos um programa em \textit{Flex} de modo a gerar um grafo de dependências entre os elementos de um documento anotado em \textit{XML}.
    Para gerar o Grafo tivemos de escrever linhas do tipo \textit{Dot-graphviz}, criando assim um ficheiro com  a extensão \textit{.dot}.         

\titulo{ Estrutura de Dados }

    Começamos por definir uma estrutura de dados capaz de guardar todas as \textit{tags} do ficheiro \textit{XML} que continham dependências de modo a construir o grafo corretamente.

\bigskip

\begin{verbatim}
typedef struct doc {
    Lista* tag_principal;       
}Document;
\end{verbatim}

    Esta é a estrutura capaz de guardar o apontador para o inicio da lista de \textit{tags} principais, importante para a criação do grafo de dependências. Esta estrutura possui só um array de apontadores.  
 

\begin{verbatim}
typedef struct lista{
    char* principal;
    struct array* filho;
    struct lista* next;
}Lista;
\end{verbatim}

    Esta é a estrutura principal da nossa implementação, contêm a \textit{tag} principal, ou seja, guarda a \textit{tag} que possui sub-elementos, contêm ainda uma estrutura auxiliar explicada a seguir e o apontador para a próxima posição da lista.
    
    

\begin{verbatim}
typedef struct array{
    char* tag;
    struct array* next;
}Array;
\end{verbatim}
    
    Esta estrutura contêm todos os sub-elementos relativos a \textit{tag} principal em causa, ou seja, tem o nome da \textit{tag} e o apontador para a próxima posição da tabela.

\begin{verbatim}
// cria uma nova estrutura capaz de armazenar uma tag secundária.
Array* newSecundaryTag(char* nome);

//cria uma nova estrutura capaz de armazenar uma tag principal.
Lista* newPrincipalNode(char* nome);

//inicia a estrutura com os apontadores para a Lista de tags principais.
Document* initDoc();

//adiciona uma nova tag principal.
int adicionaPrincipalNode(Document* doc,char* nome);

//adiciona uma nova tag secundária.
int adicionaSecudaryNode(Document* doc, char* principal, char* nome);

//verifica se a tag principal já está armazenada.
int verificaPrincipal(Document* doc, char* nome);

//verifica se a tag secundária já está armazenada.
int verificaSecundario(Document* doc, char* principal, char* nome);
\end{verbatim}

    Apresentamos aqui as funções importantes para manipular as estruturas criadas.

\begin{verbatim}
int imprimeLista(Document* doc){
    int fd = open("graph.dot",O_CREAT|O_WRONLY,0666);
    write(fd,"digraph tp2{",strlen("digraph tp2{"));
    write(fd,"\n",1);
    Lista* lista = doc->tag_principal;
    while(lista){
        Array* array = lista->filho;
        while(array){
            write(fd,lista->principal,strlen(lista->principal));
            write(fd,"->",strlen("->"));
            write(fd,array->tag,strlen(array->tag));
            write(fd,"\n",1);
            array = array->next;
        }
        lista = lista->next;
    }
    write(fd,"}",1);
    close(fd);
    return 0;
}
\end{verbatim}
    
    Salientamos esta função,visto ser, a função que trata da criação e da escrita no ficheiro \textit{.dot}. Esta trata de percorrer a estrutura depois de criada de modo a imprimir de acordo com as regras de um ficheiro \textit{.dot} e criando assim o grafo a apresentar.

\titulo{Filtagem da Informação}

    Neste capítulo apresentamos como filtramos o texto de modo a guardar a informação pretendida nas estruturas criadas.
    Vamos mostrar e explicar as codinções de contexto e expressões regulares utilizadas.


\bigskip

\begin{verbatim}
%x TAG FECHO_TAG TAG_QUANDO
\end{verbatim}
    
    Esta instrução contêm o nome das condições de contexto utilizadas, importante para o  funcionamento das mesmas. 

\begin{verbatim}
<INITIAL>{
[a-zA-Z0-9]+         {;}
\<[?].*[?]\>         {;}
\<[!].*\>            {;}
\<quando.*\/         {BEGIN TAG_QUANDO;}
\<obs\/\>            {;}
\<                   {BEGIN TAG;}
\<\/                 {BEGIN FECHO_TAG;}
}
\end{verbatim}

    Esta condição de contexto serve para filtrar tudo o que não pretendemos apanhar e iniciar as condições de contexto em que guardamos as \textit{tags} principais e secundárias. Em seguida temos a explicação de cada expressão regular. 

\begin{verbatim}
[a-zA-Z0-9]+   {;}
\end{verbatim}
    
    Com esta expressão regular conseguimos eliminar todo o texto no inicio da linha, ou seja, linhas que contenham texto antes de qualquer \textit{tag} ou só texto na linha.
    
\begin{verbatim}
\<[?].*[?]\>   {;}
\end{verbatim}

    Com esta expressão regular eliminamos todas as \textit{tags} que contenham o caracter \textbf{\?}, ou seja são removidas todas as instruções de processamento, visto não interessarem para o nosso grafo.

\begin{verbatim}
\<[!].*\>      {;}
\end{verbatim}

    Por último eliminamos todos os comentários que possam surgir no ficheiro visto também não terem interesse. 

\begin{verbatim}
\<quando.*\/   {BEGIN TAG_QUANDO;}
\end{verbatim}
    
    Quando apanhamos uma \textit{tag} começada por \textbf{quando} damos inicio a condição de contexto evidenciada \textbf{TAG\_QUANDO} e explicada posteriormente.

\begin{verbatim}
\<obs\/\>      {;}
\end{verbatim}

    Esta expressão regular é referente ao ficheiro \textbf{processos.xml} visto quando a \textit{tag} \textbf{obs}, quando não possui qualquer conteúdo apresentar esta representação \textbf{\textless obs\/\textgreater} e então decidimos remover para evitar situações de erro na criação do Grafo.

\begin{verbatim}
\<   {BEGIN TAG;}
\end{verbatim}
    
    Quando é apanhado um sinal de \textbf{\textless} começamos a condição de contexto \textbf{TAG} explicada posteriormente.

\begin{verbatim}
\<\/    {BEGIN FECHO_TAG;}
\end{verbatim}

    Por último neste bloco inicial, iniciamos a condição de contexto \textbf{FECHO\_TAG}, quando apanhamos uma \textit{tag} de fecho do ficheiro \textbf{XML}.


\begin{verbatim}
<TAG>{
[a-zA-Z0-9]+                   {tag = strdup(yytext);}
[a-zA-Z0-9]+?[_][a-zA-Z0-9]+   {tag = strdup(yytext);}
[ ].*[=].*\>{   
                                                   
    hierarquia[nivel] = strdup(tag);
    nivel++;
    adicionaPrincipalNode(doc,tag);
    BEGIN INITIAL;
}
\>{
    if(strcmp(tag,"obs")==0 || strcmp(tag,"pai")==0 || strcmp(tag,"mae")==0  ){
    BEGIN INITIAL;
    }else{
        hierarquia[nivel] = strdup(tag);
        nivel++;
        adicionaPrincipalNode(doc,tag);
        BEGIN INITIAL;
    }
}
\>.*\<\/.*\>{
        adicionaSecudaryNode(doc,hierarquia[nivel-1],tag);
        BEGIN INITIAL;
    }
}
\end{verbatim}

    Depois do bloco inicial, temos a condição de contexto \textbf{TAG}, vai ser neste bloco que vamos começar a adicionar as \textit{tags} a estrutura criada.

\begin{verbatim}
[a-zA-Z0-9]+                   {tag = strdup(yytext);}
[a-zA-Z0-9]+?[_][a-zA-Z0-9]+   {tag = strdup(yytext);}
\end{verbatim}

    Estas expressões regulares são para apanhar os nomes das \textit{tags} e por sua vez guardar esse nome numa vaŕiável.


\begin{verbatim}
[ ].*[=].*\> {   
                                                   
    hierarquia[nivel] = strdup(tag);
    nivel++;
    adicionaPrincipalNode(doc,tag);
    BEGIN INITIAL;
}
\end{verbatim}

    Sabendo a partida que as \textit{tags} com atributos são \textit{tags} principais guardamos na estrutura sempre que as apanhamos. Usando uma estrutura auxiliar chamada \textbf{hierarquia} para saber em que nível se encontra de modo a gerarmos as dependências corretamente.

\begin{verbatim}
\> {
        if(strcmp(tag,"obs")==0 || strcmp(tag,"pai")==0 || strcmp(tag,"mae")==0  ){
        BEGIN INITIAL;
   }else{
        hierarquia[nivel] = strdup(tag);
        nivel++;
        adicionaPrincipalNode(doc,tag);
        BEGIN INITIAL;
    }
}
\end{verbatim}

    Aqui guardamos todas as \textit{tags} principais que não possuem atributos, mantendo novamente a informação quanto a sua posição na hierarquia.

\begin{verbatim}
\>.*\<\/.*\> {
        adicionaSecudaryNode(doc,hierarquia[nivel-1],tag);
        BEGIN INITIAL;
    }
}
\end{verbatim}
    
    Nesta expressão regular captamos todas as \textit{tags} secundárias, relativas a \textit{tag} principal captada anteriormente e guardada no \textit{array} hierarquia no nível anterior ao que estamos colocados.



\begin{verbatim}
<FECHO_TAG>{
[a-zA-Z0-9]+                   {tag = strdup(yytext);}
[a-zA-Z0-9]+?[_][a-zA-Z0-9]+   {tag = strdup(yytext);} 
\>  {
        if(strcmp(tag,"obs")==0){
        BEGIN INITIAL;
        }else{
            nivel--;
            free(hierarquia[nivel]);
            if(nivel!=0){
                adicionaSecudaryNode(doc,hierarquia[nivel-1],tag);
            }
            BEGIN INITIAL;
        }
    }
}
\end{verbatim}

    Nesta condição de contexto fazemos o tratamento quanto ao fecho de \textit{tags} do ficheiro.

\begin{verbatim}
\>  {
        if(strcmp(tag,"obs")==0){
        BEGIN INITIAL;
        }else{
            nivel--;
            free(hierarquia[nivel]);
            if(nivel!=0){
                adicionaSecudaryNode(doc,hierarquia[nivel-1],tag);
            }
            BEGIN INITIAL;
        }
    }
\end{verbatim}

    Quando uma \textit{tag} fecha é adicionado como \textit{tag} secundária a \textit{tag} que está acima da mesma na hierarquia.

\begin{verbatim}
<TAG_QUANDO>{
\>              {
                adicionaSecudaryNode(doc,hierarquia[nivel-1],"quando");
                BEGIN INITIAL;  
                }   
}
\end{verbatim}
    
    Esta condição de contexto foi adicionada devido a \textit{tag} \textbf{textless quando textgreater} presente no ficheiro \textbf{legenda.xml}.


\begin{verbatim}
int main(){
    doc = initDoc();
    tag = NULL;
    int i;
    for(i=0;i<20;i++){
        hierarquia[i] = NULL;
    }
    nivel=0;
    yylex();
    imprimeLista(doc);
    return 0;
}
\end{verbatim}
    
    Esta é a nossa função \textbf{main} que trata de iniciar a estrutura e o \textbf{array} hierarquia e imprimir para o ficheiro.



\chapter{Conclusão} \label{concl}

    Durante a realização deste trabalho prático, que foi o segundo desta unidade curricular, várias foram as etapas que tivemos de passar para chegar ao resultado final.
    Em comparaão com o trabalho anterior achamos que este foi mais complexo e trabalhoso, mas ao mesmo tempo foi bastante importante, visto que a sua elaboração permitiu melhorar a nossa capacidade de escrever Expressões Regulares (ER) e, a partir destas, desenvolver Processadores de Linguagem Regulares, tendo em vista a filtragem e transformação de textos.
    Para isto foi necessário aprender e desenvolver a nossa abilidade de gerar filtros de texto em \textbf{FLex} e assim desenvolver a solução necessária para resolver o enunciado escolhido.
    Infelizmente, ao contrário do trabalho anterior, não conseguimos resolver nenhum enunciado extra, devido a falta de tempo, para a resolução do mesmo.
    Concluindo, no geral estamos satisfeitos com o trabalho desenvolvido, tendo este se relevado bastante útil no aperfeiçoamento das nossas habilidades no que toca a gerar filtros de texto em \textbf{FLex}.
    
\bibliographystyle{alpha}
\bibliography{relprojLayout}

\bibliography{relprojLayout}



\end{document} 